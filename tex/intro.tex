\chapter{간단한 소개}
아직 아는것이 많지 않기에, 아는것이 생긴다면 시간이 되는대로 업데이트
하겠다. 일단은 지금까지 시행착오도 많이 거치며 알아낸 것들을 정리하기 위해서 이
문서를 쓴다. 나중에 어떻게 될지는 모르겠지만, 가능한 나보다 늦게 시뮬레이션을
시작하는 사람이 나보다 적은 시간을 들여 내가 알아낸 것들을 알 수 있도록 쓸
것이다.

\vspace{5mm}
GEANT는 Geometry ANd Tracking의 약자로, 뒤에있는 4는 당연히 4번째
버전임을 나타낸다. GEANT4는 컴퓨터 안에서 가상의 실험실을 만들고, 그 안에 여러
검출기를 놓아두고 빔(입자 혹은 광자)을 이용해서 실험하는 것을 도와주는
\emph{Library Package}이다. 우리가 해야할 것은 이 패키지에서 적절한 것을 꺼내와
적절하게 배치하고, 실행시켜주는것이다.

라이브러리 패키지가 뭔지 잘 다가오지 않는 나같은 사람이 있을지도 몰라 간단히
설명해보도록 하겠다. 아무것도 없는 상태에서 시뮬레이션을 하려고 한다고 해보자.
먼저 기본 단위를 정해야 한다. 길이, 시간, 에너지, 전하량 등등. 계산을 좀 편하게
하기 위해서, 숫자 세개의 묶음이 벡터라는것도 알려줘야 하고, 이 벡터들 사이에
연산 방법을 알려줘야 한다. 속도를 알려주면, 이 속도라는게 정해준 단위 시간당
얼마만큼의 거리를 프로그램내에서 진행한다는 것을 알려줘야 한다. 검출기나 기타
물질들과 어떤 상호작용을 하는지 알려줘야 한다. 이처럼 여러가지 해야할 것들을
하나의 패키지로 묶어서 사용할 수 있게 해놓은것이 라이브러리 패키지이다.

\vspace{5mm}
\emph{참고}: 이 문서는 GEANT4.9.4.p01버전과 CLHEP\_2.1.0.1버전을 이용해서
만들어졌다. 나중에 GEANT4가 업데이트 되면서 실행이 안되는 부분이 생길 수도
있다.
