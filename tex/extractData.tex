\chapter{데이타 뽑아내기}

\setcounter{code}{0}

데이타를 뽑아내기 위해서는 4개의 파일이 더 필요하다.
\begin{verbatim}
- exampleDetectorROGeometry.cc
- DetHit.cc
- DetSD.cc
- exampleEventAction.cc
\end{verbatim}

\section{DetectorROGeometry}
\texttt{RO}는 ReadOut의 약자로, 검출기에서 검출된 신호를 읽어내는 부분을 정의하는
파일이다. 굳이 이렇게 따로 구조를 만들지 않고, DetectorConstruction에서
검출기를 여러개로 나누어서 배치하는 방법도 있지만, 그렇게 하면 메모리도 많이
먹고, 검출기 구조를 화면에 출력해서 볼 때 속도도 많이 느리다. 우리 예제에서는
별로 상관이 없지만, 검출기 구조가 매우 복잡한 경우, ROGeometry를 검출기 전체를
덮게 한 다음, 검출기 부분에 저장된 에너지만 빼오게 할 수 있으므로 시뮬레이션이
더 간편해 질 수 있다. 구조는 DetectorConstruction과 거의 같으므로, 간단하게
설명만 하고 가겠다.

헤더 파일의 기본구조는 Code~\ref{code-4-1}과 같고, 다 만들어진 코드는
Code~\ref{code-4-3}과 같다. 마찬가지로 소스파일은 Code~\ref{code-4-2},
Code~\ref{code-4-4} 그리고 Code~\ref{code-4-5}에 있다. 클래스 생성자 괄호 안의 인수들은,
DetectorConstruction에서 ROGeometry를 불러올 때 만들어진 검출기와 동일한 크기의
ROGeometry를 만들기 위해 넣었다. 기본으로 필요한 함수는
\texttt{G4VPhysicalVolume* Build()}이다. ROGeometry를 다 만든 후,
DetectorConstruction에 Code~\ref{code-3-9}기준으로 7에서 9번줄 사이에
\begin{pc}
\begin{lstlisting}
#include "exampleDetectorROGeometry.hh"
\end{lstlisting}
\end{pc}
를 넣어주어 ROGeometry 클래스를 불러오고, 같은 코드의 39에서 40번째 줄 사이에
\begin{pc}
\begin{lstlisting}
exampleDetectorROGeometry* detROGeometry
                         = new exampleDetectorROGeometry("DetROGeometry",
                                                         labX, labY, labZ,
                                                         detX, detY, detZ,
                                                         G4ThreeVector(0, 0, 5*cm));

detROGeometry -> BuildROGeometry();
\end{lstlisting}
\end{pc}
를 넣어주면, 자동으로 \texttt{Build()}함수를 불러와서 넘겨준 인수들을 가지고
ROGeometry를 만든다.

\begin{code}[p]
\begin{lstlisting}
  1 #ifndef EXAMPLEDETECTORROGEOMETRY_h
  2 #define EXAMPLEDETECTORROGEOMETRY_h 1
  3 
  4 #include "G4VReadOutGeometry.hh"
  5 #include "G4ThreeVector.hh"
  6 
  7 class exampleDetectorROGeometry: public G4VReadOutGeometry
  8 {
  9   public:
 10     exampleDetectorROGeometry(G4String aString);
 11     ~exampleDetectorROGeometry();
 12     
 13   private:
 14     G4VPhysicalVolume* Build();
 15 };  
 16 #endif
\end{lstlisting}
\caption{\texttt{exampleDetectorROGeometry.hh} (Skeleton) \label{code-4-1}}
\end{code}

\begin{code}[p]
\begin{lstlisting}
  1 #include "exampleDetectorROGeometry.hh"
  2 
  3 #include "G4Box.hh"
  4 #include "G4ThreeVector.hh"
  5 #include "G4LogicalVolume.hh"
  6 #include "G4PVPlacement.hh"
  7 #include "G4PVReplica.hh"
  8 #include "G4Material.hh"
  9 
 10 exampleDetectorROGeometry::exampleDetectorROGeometry(G4String aString)
 11 :G4VReadOutGeometry(aString)
 12 {
 13 }
 14 
 15 exampleDetectorROGeometry::~exampleDetectorROGeometry()
 16 {
 17 }
 18 
 19 G4VPhysicalVolume* exampleDetectorROGeometry::Build()
 20 {
 21 } 
\end{lstlisting}
\caption{\texttt{exampleDetectorROGeometry.cc} (Skeleton) \label{code-4-2}}
\end{code}

\begin{code}[p]
\begin{lstlisting}
  1 #ifndef EXAMPLEDETECTORROGEOMETRY_h
  2 #define EXAMPLEDETECTORROGEOMETRY_h 1
  3 
  4 #include "G4VReadOutGeometry.hh"
  5 #include "G4ThreeVector.hh"
  6 
  7 class exampleDetectorROGeometry: public G4VReadOutGeometry
  8 {
  9   public:
 10     exampleDetectorROGeometry(G4String aString,
 11                               G4double,
 12                               G4double,
 13                               G4double,
 14                               G4double,
 15                               G4double,
 16                               G4double,
 17                               G4ThreeVector);
 18 
 19     ~exampleDetectorROGeometry();
 20 
 21   private:
 22     G4VPhysicalVolume* Build();
 23 
 24   private:
 25     G4double labX, labY, labZ;
 26     G4double detX, detY, detZ;
 27     G4ThreeVector detPosition;
 28 };
 29 #endif
\end{lstlisting}
\caption{\texttt{exampleDetectorROGeometry.hh} (Complete) \label{code-4-3}}
\end{code}

\begin{code}[p]
\begin{lstlisting}
  1 #include "exampleDetectorROGeometry.hh"
  2 
  3 #include "G4Box.hh"
  4 #include "G4ThreeVector.hh"
  5 #include "G4LogicalVolume.hh"
  6 #include "G4PVPlacement.hh"
  7 #include "G4PVReplica.hh"
  8 #include "G4Material.hh"
  9 
 10 #include "DetDummySD.hh"
 11 
 12 exampleDetectorROGeometry::exampleDetectorROGeometry(G4String aString,
 13                                                      G4double val1,
 14                                                      G4double val2,
 15                                                      G4double val3,
 16                                                      G4double val4,
 17                                                      G4double val5,
 18                                                      G4double val6,
 19                                                      G4ThreeVector val7)
 20 :G4VReadOutGeometry(aString),
 21  labX(val1), labY(val2), labZ(val3),
 22  detX(val4), detY(val5), detZ(val6),
 23  detPosition(val7)
 24 {
 25 }
 26 
 27 exampleDetectorROGeometry::~exampleDetectorROGeometry()
 28 {
 29 }
 30 
 31 G4VPhysicalVolume* exampleDetectorROGeometry::Build()
 32 {
 33   G4Material* dummyMat = new G4Material("dummyMat", 1., 1*g/mole, 1*g/cm3);
 34 
 35   // make laboratory
 36   G4Box* ROLabSolid = new G4Box("ROLabSolid", labX/2, labY/2, labZ/2);
 37   G4LogicalVolume* ROLabLV = new G4LogicalVolume(ROLabSolid, dummyMat, "ROLabLV");
 38   G4PVPlacement* ROLabPV = new G4PVPlacement(0, G4ThreeVector(), "ROLabPV",
                                                 ROLabLV, 0, false, 0);
 39 
 40   const G4int numOfXVoxels = 15;
 41   const G4int numOfYVoxels = 15;
 42   const G4int numOfZVoxels = 300;
 43 
 44   // make detector
 45   G4Box* RODetSolid = new G4Box("ROTPCSolid", detX/2, detY/2, detZ/2);
 46   G4LogicalVolume* RODetLV = new G4LogicalVolume(RODetSolid, dummyMat, "RODetLV");
 47   G4VPhysicalVolume* RODetPV = new G4PVPlacement(0, detPosition, "RODetPV",
                                                     RODetLV, ROLabPV, false, 0);
\end{lstlisting}
\caption{\texttt{exampleDetectorROGeometry.cc} (Complete - 1) \label{code-4-4}}
\end{code}

\begin{code}[p]
\begin{lstlisting}
 48
 49   // make voxel
 50   G4Box* RODetXVoxelSolid = new G4Box("RODetXVoxelSolid",
 51                                       detX/numOfXVoxels/2,
 52                                       detY/2,
 53                                       detZ/2);
 54   G4LogicalVolume* RODetXVoxelLV = new G4LogicalVolume(RODetXVoxelSolid,
                                                           dummyMat,
                                                           "RODetXVoxelLV");
 55   G4VPhysicalVolume* RODetXVoxelPV = new G4PVReplica("RODetXVoxelPV",
 56                                                      RODetXVoxelLV,
 57                                                      RODetPV,
 58                                                      kXAxis,
 59                                                      numOfXVoxels,
 60                                                      detX/numOfXVoxels);
 61 
 62   G4Box* RODetYVoxelSolid = new G4Box("RODetYVoxelSolid",
 63                                       detX/numOfXVoxels/2,
 64                                       detY/numOfYVoxels/2,
 65                                       detZ/2);
 66   G4LogicalVolume* RODetYVoxelLV = new G4LogicalVolume(RODetYVoxelSolid,
                                                           dummyMat,
                                                           "RODetYVoxelLV");
 67   G4VPhysicalVolume* RODetYVoxelPV = new G4PVReplica("RODetYVoxelPV",
 68                                                      RODetYVoxelLV,
 69                                                      RODetXVoxelPV,
 70                                                      kYAxis,
 71                                                      numOfYVoxels,
 72                                                      detY/numOfYVoxels);
 73 
 74   G4Box* RODetZVoxelSolid = new G4Box("RODetZVoxelSolid",
 75                                       detX/numOfXVoxels/2,
 76                                       detY/numOfYVoxels/2,
 77                                       detZ/numOfZVoxels/2);
 78   G4LogicalVolume* RODetZVoxelLV = new G4LogicalVolume(RODetZVoxelSolid,
                                                           dummyMat,
                                                           "RODetZVoxelLV");
 79   new G4PVReplica("RODetZVoxelPV",
 80                   RODetZVoxelLV,
 81                   RODetYVoxelPV,
 82                   kZAxis,
 83                   numOfZVoxels,
 84                   detZ/numOfZVoxels);
 85 
 86   DetDummySD *dummySD = new DetDummySD();
 87   RODetZVoxelLV -> SetSensitiveDetector(dummySD);
 88 
 89   return ROLabPV;
 90 }
\end{lstlisting}
\caption{\texttt{exampleDetectorROGeometry.cc} (Complete - 2) \label{code-4-5}}
\end{code}

Code~\ref{code-4-4}와 Code~\ref{code-4-5}를 보면, 전반적인 구조는 아까 말했듯이
DetectorConstruction과 비슷하다. \texttt{G4PVReplica}가 사용됐는데, 이 클래스는
똑같은 크기의 부피를 정해준 축을 따라 정해준 개수만큼 복사하는 역할을 한다.
코드를 보면서 보자. 
\begin{pc}
\begin{lstlisting}
G4VPhysicalVolume* RODetXVoxelPV = new G4PVReplica("RODetXVoxelPV",
                                                   RODetXVoxelLV,
                                                   RODetPV,
                                                   kXAxis,
                                                   numOfXVoxels,
                                                   detX/numOfXVoxels);
\end{lstlisting}
\end{pc}
이 코드는 \texttt{RODetXVoxelPV}라는 이름을 가지고, \texttt{RODetXVoxelLV}를
\texttt{RODetPV}부피 안에서 $x$축을 따라 \texttt{numOfXVoxels} 갯수만큼
만드는데, 각각의 부피 중심 사이간격은 \texttt{detX/numOfXVoxels}이다. 같은
방식으로 나머지도 이해하면 된다. 큰 박스 하나를 $x$축을 따라 자르고, 이 $x$축을
따라 자른 박스를 $y$축을 따라 다시 자른 후, 이 길다란 막대를 $z$축을 따라
잘라서 조그마한 상자른 만든다. 마지막으로 이렇게 자른 모든께 들어가 있는
실험실의 Physical Volume을 반환해주는걸 잊으면 안된다.
\begin{pc}
\begin{lstlisting}
return ROLabPV;
\end{lstlisting}
\end{pc}
Code~\ref{code-4-5}를 보면, 86과 87줄에 
\begin{pc}
\begin{lstlisting}
 86   DetDummySD *dummySD = new DetDummySD;
 87   RODetZVoxelLV -> SetSensitiveDetector(dummySD);
\end{lstlisting}
\end{pc}
요런게 있는데, 클래스 이름에서 볼 수 있듯이 Dummy다. 우리가 나눈 Voxel를
Sensitive하게 활성화만 시켜주어 나눠진 부분을 인식시키는 역할을 한다.
소스파일은 필요가 없고, 헤더파일은 Code~\ref{code-4-6}에서 볼 수 있다.

\begin{code}[p]
\begin{lstlisting}
  1 #ifndef DETDUMMYSD_h
  2 #define DETDUMMYSD_h 1
  3 
  4 #include "G4VSensitiveDetector.hh"
  5 
  6 class G4Step;
  7 
  8 class DetDummySD : public G4VSensitiveDetector
  9 {
 10 public:
 11   DetDummySD() {};
 12   ~DetDummySD() {};
 13 
 14   void Initialize(G4HCofThisEvent* ) {};
 15   G4bool ProcessHits(G4Step* ,G4TouchableHistory*) {return false;}
 16   void EndOfEvent(G4HCofThisEvent*) {};
 17   void clear() {};
 18   void DrawAll() {};
 19   void PrintAll() {};
 20 };
 21 
 22 DetDummySD::DetDummySD() : G4VSensitiveDetector("dummySD")
 23 {}
 24 #endif
\end{lstlisting}
\caption{\texttt{DetDummySD.hh} (Complete) \label{code-4-6}}
\end{code}

\section{DetHit}
이 파일은 한개의 Hit을 담을 묶음을 만드는 역할을 한다.
헤더파일(Code~\ref{code-4-6})은 한개의 Hit과 다른 Hit을 비교하는 코드, Hit에
대한 메모리를 할당하는 코드, 만들 Hit을 이용해서 새로운 Hit을 만드는 코드,
Hit에 대한 정보를 요구할 때 정보를 넘겨주는 코드 등등이 있는데, 나중에 여러개의
검출기에서 서로 다른 Hit을 받아낼 때 코드를 참고해서 잘 수정하면 된다. 잠깐 보면
뭘 바꿔야 하는지 알 수 있으리라.  소스파일(Code~\ref{code-4-7})은 외부에서 Hit을
받아서 내부의 변수로 저장해주는 코드를 담고있다.

\begin{code}[p]
\begin{lstlisting}
  1 #ifndef DETHIT_h
  2 #define DETHIT_h 1
  3 
  4 #include "G4ThreeVector.hh"
  5 #include "G4VHit.hh"
  6 #include "G4THitsCollection.hh"
  7 #include "G4Allocator.hh"
  8 
  9 class DetHit: public G4VHit
 10 {
 11   private:
 12     G4int voxelX, voxelY, voxelZ;
 13     G4double energyDeposit;
 14 
 15   public:
 16     DetHit();
 17     DetHit(G4int, G4int, G4int, G4double);
 18     virtual ~DetHit();
 19 
 20     DetHit(const DetHit &right);
 21     const DetHit &operator=(const DetHit &right);
 22 
 23     // new/delete operators
 24     void *operator new(size_t);
 25     void operator delete(void *aHit);
 26 
 27     // set/get functions
 28     G4int GetVoxelX() const { return voxelX; };
 29     G4int GetVoxelY() const { return voxelY; };
 30     G4int GetVoxelZ() const { return voxelZ; };
 31     G4double GetEdep() const { return energyDeposit; };
 32 };
 33 
 34 inline DetHit::DetHit(const DetHit &right)
 35 :G4VHit()
 36 {
 37   voxelX = right.voxelX;
 38   voxelY = right.voxelY;
 39   voxelZ = right.voxelZ;
 40   energyDeposit = right.energyDeposit;
 41 }
 42 
 43 inline const DetHit &DetHit::operator=(const DetHit &right)
 44 {
 45   voxelX = right.voxelX;
 46   voxelY = right.voxelY;
 47   voxelZ = right.voxelZ;
 48   energyDeposit = right.energyDeposit;
 49   return *this;
 50 }
 51 
 52 extern G4Allocator<DetHit> DetHitAllocator;
 53 
 54 inline void *DetHit::operator new(size_t)
 55 {
 56   void *aHit = (void *)DetHitAllocator.MallocSingle();
 57   return aHit;
 58 }
 59 
 60 inline void DetHit::operator delete(void *aHit)
 61 {
 62   DetHitAllocator.FreeSingle((DetHit *) aHit);
 63 }
 64 #endif
\end{lstlisting}
\caption{\texttt{DetHit.hh} (Complete) \label{code-4-7}}
\end{code}

\begin{code}[p]
\begin{lstlisting}
  1 #include "DetHit.hh"
  2 
  3 // allocator
  4 G4Allocator<DetHit> DetHitAllocator;
  5 
  6 DetHit::DetHit()
  7 :voxelX(0), voxelY(0), voxelZ(0), energyDeposit(0)
  8 {
  9 }
 10 
 11 DetHit::DetHit(G4int x, G4int y, G4int z, G4double edep)
 12 :voxelX(x), voxelY(y), voxelZ(z), energyDeposit(edep)
 13 {
 14 }
 15 
 16 DetHit::~DetHit()
 17 {
 18 }
\end{lstlisting}
\caption{\texttt{DetHit.cc} (Complete) \label{code-4-7}}
\end{code}

\section{DetSD}
DetSD의 헤더파일은 Code~\ref{code-4-8}에서 볼 수 있다. 이게 기본이면서
최종버전이다. 검출기가 여러개라면, 여러개의 SD가 필요하므로, 중간에 몇몇개
바꿔야 할 것들만 눈여겨 보자. DetHit에서 볼 수 있듯이 우리가 이번에 사용한
검출기를 Det라고 두었기 때문에 DetHit이 되고, DetSD가 된 것이다.
\texttt{ProcessHits}함수와 그 함수의 인수 \texttt{G4Step *aStep}에서 볼 수
있듯이, 한 스텝에서 발생한 Hit를 받아서 \texttt{Initialize}에서 초기화해 준
\texttt{HCTE}(\texttt{H}it \texttt{C}ollection of \texttt{T}his
\texttt{E}vent)에 저장하는 역할을 하는 파일이다.

Code~\ref{code-4-9}에서 \texttt{DetSD}가 받는 인수 \texttt{name}은 나중에 여러
검출기를 작성했을 때, GEANT4와 상호작용할 수 있는 커맨드라인에서 아스키
트리형태로 Sensitive Detector로 등록된 검출기들의 리스트를 보여줄 때
사용한다. Code~\ref{code-4-9}를 보면, \texttt{Initialize}에서 Hit들의 컬렉션을
만들고, 이것을 \texttt{HCTE}에 등록시켜주는걸 볼 수 있다.

\texttt{ProcessHits}에서는 한 스텝에서 저장된 에너지가 없거나,
\texttt{ROHist}가 없으면 다른 작업을 수행하지 않고 바로 끝나버리게 한 것을 볼
수 있다. \texttt{ROHist}에는 우리가 ROGeometry에서 나눠주었던 Voxel에 대한
정보가 들어있다. 그 다음, Hit이 일어난 부피의 이름이 우리가
DetectorContruction에서 만들어준 \texttt{detPV} 일 때에만 Hit을 저장하게 만드는
코드를 볼 수 있다.

\begin{code}[p]
\begin{lstlisting}
  1 #ifndef DETSD_h
  2 #define DETSD_h 1
  3 
  4 #include "G4VSensitiveDetector.hh"
  5 #include "DetHit.hh"
  6 
  7 class G4HCofThisEvent;
  8 class G4Step;
  9 class G4TouchableHistory;
 10 
 11 class DetSD: public G4VSensitiveDetector
 12 {
 13   private:
 14     G4THitsCollection<DetHit> *hitsCollection;
 15 
 16   public:
 17     DetSD(const G4String& name);
 18     virtual ~DetSD();
 19 
 20     virtual G4bool ProcessHits(G4Step *aStep, G4TouchableHistory *ROHist);
 21     virtual void Initialize(G4HCofThisEvent *HCTE);
 22     virtual void EndOfEvent(G4HCofThisEvent *HCTE);
 23 };
 24 #endif
\end{lstlisting}
\caption{\texttt{DetSD.hh} (Complete) \label{code-4-8}}
\end{code}

\begin{code}[p]
\begin{lstlisting}
  1 #include "G4VPhysicalVolume.hh"
  2 #include "G4Step.hh"
  3 #include "G4Track.hh"
  4 #include "G4TouchableHistory.hh"
  5 #include "G4SDManager.hh"
  6 #include "DetSD.hh"
  7 #include "DetHit.hh"
  8 
  9 DetSD::DetSD(const G4String &name)
 10 :G4VSensitiveDetector(name)
 11 {
 12   collectionName.insert("DetHitsCollection");
 13 }
 14 
 15 DetSD::~DetSD()
 16 {
 17 }
 18 
 19 void DetSD::Initialize(G4HCofThisEvent *HCTE)
 20 {
 21   hitsCollection
      = new G4THitsCollection<DetHit>(SensitiveDetectorName,
                                      collectionName[0]);
 22 
 23   G4int hcid
       = G4SDManager::GetSDMpointer() -> GetCollectionID(collectionName[0]);
 24   HCTE -> AddHitsCollection(hcid, hitsCollection);
 25 }
 26 
 27 G4bool DetSD::ProcessHits(G4Step *aStep, G4TouchableHistory *ROHist)
 28 {
 29   if (!ROHist)  return false;
 30 
 31   G4double energyDeposit = aStep -> GetTotalEnergyDeposit();
 32   if (energyDeposit == 0.)  return false;
 33 
 34   if ((aStep -> GetPreStepPoint() -> GetPhysicalVolume() -> GetName()
            == "detPV") && (energyDeposit != 0.))
 35   {
 36     G4int voxelZ = ROHist -> GetReplicaNumber(0);
 37     G4int voxelY = ROHist -> GetReplicaNumber(1);
 38     G4int voxelX = ROHist -> GetReplicaNumber(2);
 39 
 40     DetHit *aHit = new DetHit(voxelX, voxelY, voxelZ, energyDeposit);
 41     hitsCollection -> insert(aHit);
 42   }
 43 
 44   return true;
 45 }
 46 
 47 void DetSD::EndOfEvent(G4HCofThisEvent *HCTE)
 48 {
 49 }
\end{lstlisting}
\caption{\texttt{DetSD.cc} (Complete) \label{code-4-9}}
\end{code}

이렇게 모든 작업을 해 준 후, DetectorConstruction에서 ROGeometry를 정의해준
부분 밑에 SD를 등록해주어야 한다. 이 때, ROGeometry와 실제 검출기 부피를 둘 다
등록해줘야 하는것에 주의하자. 당연히 DetSD 클래스를 사용했으므로, 위쪽에
헤더파일을 include해줘야 한다.
\begin{pc}
\begin{lstlisting}
// Sensitive Detector
G4SDManager* sdManager = G4SDManager::GetSDMpointer();

DetSD* detSD = new DetSD("/det");
detSD -> SetROgeometry(detROGeometry);
sdManager -> AddNewDetector(detSD);
detLV -> SetSensitiveDetector(detSD);
\end{lstlisting}
\end{pc}

모든 수정이 다 끝난 DetectorConstruction은 Code~\ref{code-4-11}에서 볼 수 있다.

\begin{code}[p]
\begin{lstlisting}
  1 #include "exampleDetectorConstruction.hh"
  2 
  3 #include "G4LogicalVolume.hh"
  4 #include "G4VPhysicalVolume.hh"
  5 #include "G4VisAttributes.hh"
  6 #include "G4Colour.hh"
  7 
  8 #include "G4Box.hh"
  9 #include "G4PVPlacement.hh"
 10 
 11 #include "DetSD.hh"
 12 #include "G4SDManager.hh"
 13 
 14 #include "exampleDetectorROGeometry.hh"
 15 
 16 exampleDetectorConstruction::exampleDetectorConstruction()
 17 {
 18   ConstructMaterials();
 19   DefineDimensions();
 20 }
 21 
 22 exampleDetectorConstruction::~exampleDetectorConstruction()
 23 {
 24   DestructMaterials();
 25 }
 26 
 27 void exampleDetectorConstruction::DefineDimensions()
 28 {
 29   labX = 20*cm;
 30   labY = 20*cm;
 31   labZ = 40*cm;
 32 
 33   detX = 15*cm;
 34   detY = 15*cm;
 35   detZ = 30*cm;
 36 }
 37 
 38 G4VPhysicalVolume* exampleDetectorConstruction::Construct()
 39 {
 40   G4Box* labSolid = new G4Box("labSolid", labX/2, labY/2, labZ/2);
 41   G4LogicalVolume* labLV = new G4LogicalVolume(labSolid, Air, "labLV");
 42   G4VPhysicalVolume* labPV = new G4PVPlacement(0, G4ThreeVector(), "labPV",
                                                   labLV, 0, false, 0);
 43 
 44   G4Box* detSolid = new G4Box("detSolid", detX/2, detY/2, detZ/2);
 45   G4LogicalVolume *detLV = new G4LogicalVolume(detSolid, Water, "detLV");
 46   new G4PVPlacement(0, G4ThreeVector(0, 0, 5*cm), "detPV", detLV, labPV,
                        false, 0);
 47 
 48   G4VisAttributes* detVisAttrib = new G4VisAttributes(G4Colour(0., 0., 1.));
 49   detLV -> SetVisAttributes(detVisAttrib);
 50 
 51   // Readout Geometry
 52   exampleDetectorROGeometry* detROGeometry
                     = new exampleDetectorROGeometry("DetROGeometry",
 53                                                  labX, labY, labZ,
 54                                                  detX, detY, detZ,
 55                                                  G4ThreeVector(0, 0, 5*cm));
 56 
\end{lstlisting}
\caption{\texttt{exampleDetectorConstruction.cc} (Complete - 1) \label{code-4-11}}
\end{code}

\begin{code}[p]
\begin{lstlisting}
 57   detROGeometry -> BuildROGeometry();
 58 
 59   // Sensitive Detector
 60   G4SDManager* sdManager = G4SDManager::GetSDMpointer();
 61 
 62   DetSD* detSD = new DetSD("/det");
 63   detSD -> SetROgeometry(detROGeometry);
 64   sdManager -> AddNewDetector(detSD);
 65   detLV -> SetSensitiveDetector(detSD);
 66 
 67   return labPV;
 68 }
 69 
 70 void exampleDetectorConstruction::ConstructMaterials()
 71 {
 72   // STP_Temperature = 0 degree Celcius
 73   const G4double labTemp = STP_Temperature + 20.*kelvin;
 74 
 75   // Elements - G4Element(name, symbol, Z-number, molecular mass)
 76   elN = new G4Element("Nitrogen", "N", 7, 14.00674*g/mole);
 77   elO = new G4Element("Oxygen", "O", 8, 15.9994*g/mole);
 78   elAr = new G4Element("Argon", "Ar", 18, 39.948*g/mole);
 79   elC = new G4Element("Carbon", "C", 6, 12.011*g/mole);
 80   elH = new G4Element("Hydrogen", "H", 1, 1.00794*g/mole);
 81 
 82   // Materials - G4Material(name, density, # of element, state, temperature)
 83   Air = new G4Material("Air", 1.2929e-03*g/cm3, 3, kStateGas, labTemp);
 84     Air -> AddElement(elN, 75.47/99.95);
 85     Air -> AddElement(elO, 23.20/99.95);
 86     Air -> AddElement(elAr, 1.28/99.95);
 87 
 88   Scint = new G4Material("Scintillator", 1.05*g/cm3, 2, kStateSolid, labTemp);
 89     Scint -> AddElement(elC, 10);
 90     Scint -> AddElement(elH, 11);
 91 
 92   Water = new G4Material("Water", 1*g/cm3, 2, kStateLiquid, labTemp);
 93     Water -> AddElement(elH, 2);
 94     Water -> AddElement(elO, 1);
 95 }
 96 
 97 void exampleDetectorConstruction::DestructMaterials()
 98 {
 99   delete Water;
100   delete Scint;
101   delete Air;
102 
103   delete elH;
104   delete elC;
105   delete elAr;
106   delete elO;
107   delete elN;
108 }
\end{lstlisting}
\caption{\texttt{exampleDetectorConstruction.cc} (Complete - 2) \label{code-4-12}}
\end{code}

\section{EventAction}

마지막으로 EventAction이다. EventAction은 이름에서 볼 수 있듯이 한 event가 있을
때마다 실행되는데, event의 시작과 끝에서 할 일을 정할 수 있다. 기본 구조이면서
최종 완성구조인 헤더파일의 코드는 Code~\ref{code-4-13}에서 볼 수 있고,
소스파일의 기본구조는 Code~\ref{code-4-14}에서 볼 수 있다.

\begin{code}[p]
\begin{lstlisting}
  1 #ifndef EXAMPLEEVENTACTION_h
  2 #define EXAMPLEEVENTACTION_h 1
  3 
  4 #include "G4UserEventAction.hh"
  5 
  6 class G4Event;
  7 
  8 class exampleEventAction: public G4UserEventAction
  9 {
 10   public:
 11     exampleEventAction();
 12     virtual ~exampleEventAction();
 13 
 14     virtual void BeginOfEventAction(const G4Event* anEvent);
 15     virtual void EndOfEventAction(const G4Event* anEvent);
 16 
 17   private:
 18     G4int hitsCollectionID;
 19 };
 20 #endif
\end{lstlisting}
\caption{\texttt{exampleEventAction.hh} (Skeleton and Complete)
\label{code-4-13}}
\end{code}

\begin{code}[p]
\begin{lstlisting}
  1 #include "G4RunManager.hh"
  2 #include "G4Event.hh"
  3 #include "G4SDManager.hh"
  4 
  5 #include "exampleEventAction.hh"
  6 
  7 exampleEventAction::exampleEventAction()
  8 {
  9 }
 10 
 11 exampleEventAction::~exampleEventAction()
 12 {
 13 }
 14 
 15 void exampleEventAction::BeginOfEventAction(const G4Event* anEvent)
 16 {
 17 }
 18 
 19 void exampleEventAction::EndOfEventAction(const G4Event* anEvent)
 20 {
 21 }
\end{lstlisting}
\caption{\texttt{exampleEventAction.cc} (Skeleton) \label{code-4-14}}
\end{code}

EventAction에서는 한 이벤트가 끝난 후, 저장돼있는 \texttt{hitCollection}을
받아와서 거기있는 정보들을 빼내어 파일로 저장하는 일을 할 것이다. 여기서는
Hit에 관한 정보를 다뤄야 하기 때문에 Code~\ref{code-4-15}에 6번째 줄에 보면
\texttt{DetHit.hh}파일을 include한 것을 볼 수 있다. 한 이벤트가 끝났을 때만 이
작업을 할 것이라, 다른곳은 손댈곳이 없고, 함수 이름에서 언제 불러와질지 예상할
수 있는 \texttt{EndOfEventAction}만을 수정할 것이다.

\begin{code}[p]
\begin{lstlisting}
  1 #include "G4RunManager.hh"
  2 #include "G4Event.hh"
  3 #include "G4SDManager.hh"
  4 
  5 #include "exampleEventAction.hh"
  6 #include "DetHit.hh"
  7 
  8 #include <fstream>
  9 #include <iomanip>
 10 
 11 exampleEventAction::exampleEventAction()
 12 {
 13 }
 14 
 15 exampleEventAction::~exampleEventAction()
 16 {
 17 }
 18 
 19 void exampleEventAction::BeginOfEventAction(const G4Event* anEvent)
 20 {
 21 }
 22 
 23 void exampleEventAction::EndOfEventAction(const G4Event* anEvent)
 24 {
 25   G4SDManager* sdManager = G4SDManager::GetSDMpointer();
 26   
 27   G4HCofThisEvent* HCTE = anEvent -> GetHCofThisEvent();
 28   if (!HCTE) return;
 29
 30   G4THitsCollection<DetHit>* HC_Det = NULL;
 31  
 32   hitsCollectionID = sdManager -> GetCollectionID("DetHitsCollection");
 33   HC_Det = (G4THitsCollection<DetHit> *)(HCTE -> GetHC(hitsCollectionID));
 34   
 35   if (HC_Det)
 36   { 
 37     G4int numHits = HC_Det -> entries();
 38     
 39     for (G4int i = 0; i != numHits; i++)
 40     { 
 41       G4int voxelX = (*HC_Det)[i] -> GetVoxelX();
 42       G4int voxelY = (*HC_Det)[i] -> GetVoxelY();
 43       G4int voxelZ = (*HC_Det)[i] -> GetVoxelZ();
 44       G4double energyDeposit = (*HC_Det)[i] -> GetEdep();
 45       
 46       std::ofstream outFile("edep.out", std::ios::app);
 47       outFile << std::setw(5) << voxelX << std::setw(5) << voxelY
                  << std::setw(5) << voxelZ << std::setw(15)
                  << energyDeposit << G4endl;
 48     }
 49   }
 50 }
\end{lstlisting}
\caption{\texttt{exampleEventAction.cc} (Complete) \label{code-4-15}}
\end{code}

시뮬레이션이 작동될 때 이미 메모리에 상주하고 있는 Sensitive Detector Manager의
포인터를 불러와서 \texttt{DetSD.cc}에서 정해주었던 hitCollection의 이름을
이용해서 \texttt{hitCollectionID}를 받아온다. 물론 이 이벤트에서 아무것도
검출이 안돼서 \texttt{HCTE}가 없다면 다음것들을 할 필요가 없으므로 끝내버리는
코드도 들어있다. Hit이 있었을 때에, 미리 만들어둔 Hit 패턴 형태의 포인터 변수
\texttt{HC\_Det}에 hitCollection내의 Hit을 차례차례 받아와서 출력하는 루프를 돈
후 프로그램이 끝난다.
