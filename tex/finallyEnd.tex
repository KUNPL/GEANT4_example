\chapter{드디어 끝!}

\setcounter{code}{0}

드디어 끝났다. 마지막으로 \texttt{example.cc}파일에 EventAction을 불러오도록 
\begin{pc}
\begin{lstlisting}
exampleEventAction* exampleEA = new exampleEventAction();
runManager -> SetUserAction(exampleEA);
\end{lstlisting}
\end{pc}
이 두줄을 PrimaryGeneratorAction뒤쪽에 추가시켜준다. 완성된 코드는
Code~\ref{code-5-1}에서 볼 수 있다.

\begin{code}[p]
\begin{lstlisting}
  1 #include "G4RunManager.hh"
  2 #include "A01PhysicsList.hh"
  3 #include "exampleDetectorConstruction.hh"
  4 #include "examplePrimaryGeneratorAction.hh"
  5 #include "exampleEventAction.hh"
  6 
  7 #include "G4UIterminal.hh"
  8 #include "G4UItcsh.hh"
  9 #include "G4UImanager.hh"
 10 
 11 #include "G4VisExecutive.hh"
 12 
 13 int main(int argc, char ** argv)
 14 {
 15   G4RunManager *runManager = new G4RunManager();
 16 
 17   exampleDetectorConstruction* exampleDC = new exampleDetectorConstruction();
 18   runManager -> SetUserInitialization(exampleDC);
 19 
 20   A01PhysicsList* physicsList = new A01PhysicsList();
 21   runManager -> SetUserInitialization(physicsList);
 22 
 23   examplePrimaryGeneratorAction* examplePGA = new examplePrimaryGeneratorAction();
 24   runManager -> SetUserAction(examplePGA);
 25 
 26   exampleEventAction* exampleEA = new exampleEventAction();
 27   runManager -> SetUserAction(exampleEA);
 28 
 29   runManager -> Initialize();
 30 
 31   G4VisManager* visManager = new G4VisExecutive();
 32   visManager -> Initialize();
 33 
 34   if (argc == 1)
 35   {
 36     G4UIsession* session = new G4UIterminal(new G4UItcsh);
 37     session -> SessionStart();
 38     delete session;
 39   }
 40   else
 41   {
 42     G4String command = "/control/execute ";
 43     G4String fileName = argv[1];
 44 
 45     G4UImanager* UI = G4UImanager::GetUIpointer();
 46     UI -> ApplyCommand(command + fileName);
 47 
 48     G4UIsession* session = new G4UIterminal(new G4UItcsh);
 49     session -> SessionStart();
 50     delete session;
 51   }
 52 
 53   delete runManager;
 54 
 55   return 0;
 56 }
\end{lstlisting}
\caption{\texttt{example.cc} (Complete) \label{code-5-1}}
\end{code}

이제 컴파일을 하고 시뮬레이션을 돌려보자. 한개의 이벤트만 주고
나온 데이타 파일을 한번 보자! 아래와 비슷하면 된거다!
\begin{verbatim}
      1     7    7    0     0.00704296
      2     7    7    0       0.121516
      3     7    7    0       0.133264
      4     7    7    0      0.0796886
      5     7    7    0     0.00617095
      6     7    7    0      0.0933267
      7     7    7    0       0.109112
      8     7    7    1       0.192196
      9     7    7    1      0.0193361
     10     7    7    1       0.150305
     11     7    7    2       0.118832
     12     7    7    2       0.327402
     13     7    7    2      0.0680953
     14     7    7    3       0.304043
     15     7    7    3       0.578538
     16     7    7    5       0.104974
     17     7    7    5      0.0766837
     18     7    7    5       0.275127
     19     7    7    6      0.0273433
     20     7    7    6      0.0327526
\end{verbatim}
순서대로 $x$, $y$, $z$의 voxel번호 그리고 저장된 에너지이다. 저장된 에너지에 숫자만
나오니까, 꼭 기본단위가 무엇인지 확인을 해야 한다.

이제 한 10000개정도의 이벤트를 만든다음 그림을 그려보면, Figure~\ref{finalResult}과
같은 그래프를 얻을 수 있다!

\vspace{5cm}
여기까지 따라오느라 수고하셨습니다!

만드느라 수고했다 나!
